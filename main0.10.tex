\documentclass[useAMS,usenatbib]{mn2e}
\usepackage{amsmath}
\usepackage{hyperref}
\hypersetup{
    colorlinks=True,
    linkcolor=red,
    filecolor=magenta,      
    urlcolor=cyan,
    citecolor=blue,
}
\usepackage{graphicx}
\usepackage{natbib}
\bibliographystyle{mn2e}
\usepackage{times}
\usepackage{float}
\usepackage{subcaption}
\usepackage{multirow}
\usepackage{color,soul}
\usepackage{import}
\usepackage[T1]{fontenc}
\usepackage{ae,aecompl}
\usepackage{amssymb}	% Extra maths symbols
\usepackage{multicol}        % Multi-column entries in tables
\usepackage{pgffor}
\usepackage{commath}	% Extra maths symbols

% 22Jan2018: following 2 commands now work (they didn't before)
\newcommand{\boldit}[1]{\textbf{\textit{#1}}}
% for changed text to be bold: uncomment following line and comment out line after that
\newcommand{\changed}[1]{\textbf{#1}} 
% for changed text to be normal: comment out line above this, uncomment line below
%\newcommand{\changed}[1]{#1}  

\newcommand \aaj {A\&A}
\newcommand \aarv {A\&ARv}%: Astronomy and Astrophysics Review (the)
\newcommand \aas{A\&AS}%: Astronomy and Astrophysics Supplement Series
\newcommand \afz {Afz}%: Astrofizika
\newcommand \aj {AJ}%: Astronomical Journal (the)
\newcommand \apss {Ap\&SS}%: Astrophysics and Space Science
\newcommand \apj {ApJ}
\newcommand \apjs {ApJS}%: Astrophysical Journal Supplement Series (the)
\newcommand \araa {ARA\&A} %: Annual Review of Astronomy and Astrophysics
\newcommand \asp {ASP Conf. Ser.}%: Astronomy Society of the Pacific Conference Series
\newcommand \azh {Azh}%: Astronomicheskij Zhurnal
\newcommand \baas {BAAS}%: Bulletin of the American Astronomical Society
\newcommand \mem {Mem. RAS}%: Memoirs of the Royal Astronomical Society
\newcommand \mnassa {MNASSA}%: Monthly Notes of the Astronomical Society of Southern Africa
\newcommand \mnras {MNRAS} %: Monthly Notices of the Royal Astronomical Society
%\newcommand {Nature}%(do not abbreviate)
\newcommand \pasj {PASJ}%: Publications of the Astronomical Society of Japan
\newcommand \pasp {PASP}%: Publications of the Astronomical Society of the Pacific
\newcommand \qjras {QJRAS}%: Quarterly Journal of the Royal Astronomical Society
\newcommand \mex {Rev. Mex. Astron. Astrofis.}%: Revista Mexicana de Astronomia y Astrofisica
%\newcommand {Science }%}%(do not abbreviate)
\newcommand \sva {SvA}%: Soviet Astronomy
\newcommand \aap {A\&A} %:Astronomy & Astrophysics
\newcommand \apjl {ApJL} %:The Astrophysical Journal Letters


\begin{document}
% TITLE
\defcitealias{Hossein12}{T12}
\defcitealias{Kinney96}{K96}
\defcitealias{Noll09}{N09}

\title[Classifying high-$z$ galaxy spectra]{Classifying galaxy spectra at $0.5<z<1$ with self-organizing maps}
%\author{rahmani.sahar}
\date{\today}
\author[S.~Rahmani, H.~Teimoorinia and P.~Barmby]{S.~Rahmani$^{1,2}$\thanks{E-mail:
srahma49@uwo.ca}, H.~Teimoorinia$^{3,4}$, P.~Barmby$^{1,2}$\\
$^{1}$Department of Physics $\&$ Astronomy, Western University, London, ON N6A 3K7, Canada\\
$^{2}$Centre for Planetary Science \& Exploration, Western University, London, ON N6A 3K7, Canada\\
$^{3}$Department of Physics $\&$ Astronomy, University of Victoria, Finnerty Road, Victoria, British Columbia, V8P 1A1, Canada\\
$^{4}$NRC Herzberg Astronomy and Astrophysics, 5071 West Saanich Road, Victoria, BC, V9E 2E7, Canada }
\maketitle


\begin{abstract}
    The spectrum of a galaxy contains information about its physical properties.
    Classifying spectra using templates helps elucidate the nature of a galaxy's energy sources.
    In this paper, we investigate the use of self-organizing maps in classifying galaxy spectra against templates.
    We trained semi-supervised self-organizing map networks using a set of
    templates covering the wavelength range from far ultraviolet to near infrared. 
    The trained networks were used to classify the spectra of a sample of 142 galaxies with $0.5 < z < 1$ and the results compared to classifications performed using K-means clustering, a supervised neural network, and chi-squared minimization.
    Spectra corresponding to quiescent galaxies were more likely to be classified similarly by all methods while starburst spectra showed more variability.
    Compared to classification using chi-squared minimization or the supervised neural network, the galaxies classed together by the self-organizing map had more similar spectra.
     The class ordering provided by the one-dimensional self-organizing maps corresponds to an ordering in physical properties, a potentially important feature for the exploration of large datasets.
\end{abstract}
\begin{keywords} 
 galaxies: high-redshift, 
 galaxies: spectra, 
 methods: observational, 
 methods: statistical, 
 methods:data analysis
\end{keywords}

\import{sections0.10/}{intro0.10.tex}

\import{sections0.10/}{data0.10.tex}

\import{sections0.10/}{method0.10.tex}

\import{sections0.10/}{results0.10.tex}

\import{sections0.10/}{summary0.10.tex}

\section*{ACKNOWLEDGMENTS}
The authors thank the referee for thorough and insightful comments which helped us to improve the work.
The authors thank S. Lianou, A. Tammour, S.C. Gallagher, R.G. Abraham, M. Daley and A. Sigut for their useful comments. 
S.R. and P.B. also acknowledge research support from the Natural Sciences and Engineering Research Council of Canada. 

\bibliographystyle{apalike}
\bibliography{ref_mining_h.bib}

\import{sections0.10/}{apps_0.10.tex}
\end{document}
