Response to Referee Report on:"Classifying galaxy spectra at $0.5<z<1$: an unsupervised approach", Sahar Rahmani et al. 
We thank the referee for her/his helpful comments and have addressed them individually below in lines beginning with "-". Changes to the text are highlighted in bold in the manuscript and also appear in quotations after our responses below. To improve clarity, we made some minor changes in the text which were not requested by the referee; these are in red bold highlighted text. 

We hope changes made to the paper, makes it more clear why SOMs is useful method for classifying galaxy spectra and how it does not consider as an arbitrary classification. 


"Most current classification methods cannot classify SEDs outside the range of templates, and will fail in special cases." (abstract)
Any classification method can generate a classification for any input.  It may be the case that this is of low confidence, or it is incorrect, but there is still some output.  I suggest rephrasing this claim to indicate the actual limitation (low accuracy? low confidence? something else?).
	- We rephrase the sentence as follows:
	    "Most current classification methods classify spectra outside the
		range of templates with low accuracy, and will fail in special cases"

In the end, how does one use the output of the SOM classification? Can you evaluate its accuracy in some way?  I understand that SOMs are exploratory and unsupervised, but there should be some demonstrable benefit or value from having done it.
    - There is no standard method to measure the accuracy of the SOM classification, however we hope that our description about comparing K-means clustering and self-organizing maps (see below and Section 4.3) sheds light on the usefulness of the SOM method. We find that, compared to K-means, SOM seems to be better at both discerning differences between similar templates and classifying objects which do not exactly match the templates.


The only statement I can see about something learned from the exercise is that the SOM yielded results with higher correlations between non-spectral properties.  But correlation is never quantified; instead plots are shown, leaving the reader's eye to estimate correlation.  To my eye, the plots in the lower row of Figure 10 have low correlation, if any.  What is the R^2 value?  And why does this matter? 
    - We added the following sentences to sections 4.1, 4.3 and 5 to elaborate on the usefulness of the SOM method:
    Section 4.1:
	    "One of the key features of SOMs is the ability to not only classify objects but to provide an {\em ordered} classification. 
       Figure~\ref{fig: props_vs_props} shows that the SOM correctly determines an ordering of the physical properties of the galaxy models used to make the classified spectra. 
       The median properties of the spectral groups change on going from one end of the SOM to the other.
       K-means clustering and supervised ANNs can sort galaxy spectra into classes but do not naturally order them."
	 
	    Section 4.3: 
	    "One of the other key features of SOMs compared to supervised classification is the ability to both interpolate between and extrapolate outside of training set classes.
        This permits a fuller exploration of dataset properties and detection of outliers."
 	    
 
	    Section 5:
        "This particular sample of high-redshift galaxies is well-described by the range of the K96 templates; some of the spectra fall in between template classes.  This implies that these galaxies are not markedly different from those found in the local universe. The self-organizing map method can order the spectra in such a way as to also match the order along the age/SFR and SSFR/M* relation.
     Other classification methods such as K-means and supervised artificial neural networks can provide a general classification with similar performance to the SOM. The ordered results and intermediate classes produced by self-organizing maps are distinct advantages of this method which can be utilized for other astronomical applications."

        - Regarding the correlation: in Fig. 9 the main message is how the colour of the points (corresponding to the classification) is ordered within each plot, rather than the actual correlation between parameters in the plot.

In response to major problems:
1.	Motivate the use of 1D versus 2D SOMs (or both).  As currently presented it is not clear what the benefit is of using either one, especially since they give different results.  Which one is better? Since they have different results, what does each one tell us?  At the end of section 4, the paper advises investigators to use the 2D approach first.  Why?  And if so, what additional knowledge is gained by applying the 1D approach? 
	- We add the following sentences to the end of section 4:"After finding and removing outliers in the input data, the use of 1D and 2D maps is complementary. One-dimensional maps are helpful for exploring the general behaviour of the data, while 2D maps show a more detailed picture."
	We also add the following sentences to section 4.3 to compare 1D and 2D maps:
    	"T12 showed that by masking emission lines (i.e., by ignoring details in spectra) two types of galaxies, Sb and SB6, can be misclassified as one another.
        As mentioned before, 1D SOMs are a good method to obtain a big picture of the problem under study and do not consider the small details. 
        Therefore, we can see that the SB6 category is placed near Sb galaxies in the 1D network in Fig.~\ref{fig: 1by22T}.
        However, in the upper right panel of Fig.~\ref{fig: 12by12}, we can see that the Sb and SB galaxies are completely separated. This shows that more details of spectral shapes are considered by 2D networks, which is one of the main advantages of  using 2D maps over 1D ones."

2. Although the paper explains why the SOM has more flexibility/expressivity than the neural network approach in reference T12, it does not compare the SOM to any other method out there.  The obvious candidate for comparison would be to apply k-means clustering (e.g., cluster the 142 spectra and then each clusters to its closest match of the original 12 templates).  But to compare the meaningfully, again you'll need some measure of performance, which currently is lacking.
    - We added Section 4.3 to the paper, which shows the result of K-means analysis on the data. 
    We classified 12 template galaxies in 4 groups and compared them with the classification from SOM. We also classified 142 galaxies using K-means clustering and matched them with the nearest K96 template using spectral angle distances. We compare the classification results with SOM classification and show their similarity and dissimilarity.

In response to minor problems:
The "mock sample" in section 3.3 does not add anything to the paper.  I suggest omitting it.  It raises more questions than it answers.  For example, is this randomly generated data?  Why 27 samples? Apparently there are only two features with only two possible values each, so there are only 4 unique possible samples, so 23 of the 27 must duplicate some other sample, and the data set is strongly over-specified.  Further, the features have values that are categorical rather than numeric (e.g., "high" and "low"), and it is not at all clear how the SOM, which employs Euclidean distance, can handle that kind of data.  Being able to divide a data set with only 4 unique samples into 4 distinct groups seems to be a non-accomplishment.
    - We removed this section. 
    
    
I could not understand how the 2D SOM was constrained so that the results in the right-hand side of Figure 11 were obtained.  This network has the same topology as the one on the left yet somehow the K96 templates "are in a small region" (p. 10, line 43, column 2).
    - We used another SOM library to create the second type of SOM network. In the second method, by controlling the number of neighbourhoods, we can squeeze all the K96 templates into a small part of the map. In this case we can check for outliers in the sample galaxies, of which none were present here.

Euclidean distance is not the best way to match spectra; I recommend using spectral angle distance instead.
    - The main advantage of the spectral angle distance over the Euclidean distance for finding differences in spectra is that in the spectral angle distance looks at the general shape of the spectrum. Although SOM uses Euclidean distance to find the differences in some layer of classification, it also considers the overall shape of the spectrum. 
    Therefore, we decided to not use the spectral angle distance for SOM method.
    However, we used spectral angle distance for the K-means clustering and for finding the best match between K-means clusters and the templates in Section 4.2. 

This isn't a fully unsupervised approach, since the SOM results are "classified" using the 12 template types (supervision).  That's not a weakness, but it means it would be more accurate to refer to the  method as "weakly supervised" or "partially supervised."
    - We add the following sentences to Section 1 where we introduced self-organizing maps: "An SOM can be trained without supervision and its results used to classify other data. Therefore, this method can be considered as (semi)-supervised." 

Section 4.1.1 (p. 6, line 37-39, column 2) states that "The black color indicates that the left neuron is completely different from the other two groups" but earlier it was explained that these colors are relative to the range of differences present.  So black and white could be quite close (e.g., 0.1 and 0.2) or they could be large (e.g., 1 and 100000) but there's no way to know from the color.
    - The referee is correct: the differences are relative and not absolute. We change the phrase to:
        "They are more similar to each other than to the ones in the left neurons". 
        
        
Section 4.1.1 (p. 7, lines 10-11, column 1) confusingly states that Figure 5 shows that "5 groups on the left-side neurons are separated from 7 groups on the right side of the map with two dark-grey colors between them."  There are indeed 5 groups then 2 dark-grey links then 7 groups, but those 7 are broken into sub groups with even greater contrast (black links).  So it would be more appropriate to say that the galaxies are split as 5 - 1 - 1 - 5 or even 6 - 1 - 6.
    - We changed the lines to the following:
        "Five groups in the left-side neurons are separated from the rest of the groups with two dark-grey colours between them. Five groups in the right-side neurons are separated from the rest by the black colour, and Sc and SB6 galaxies are in two separate blocks. This figure is a good example of the problem of using one-dimensional SOMs to explore the data. Since are the templates are forced to be arranged in a linear array, the SB6 and Sc groups are isolated and not connected to any other groups. However, in a two-dimensional map, due to the connection of each neuron to 6 more neurons, template galaxies have more degrees of freedom to find their neighbourhood.

Section 4.1.2 (p. 7, line 9, column 2) says that Figure 8 has 12 plots, but actually it has 14.
    - Changed it to 14.

It is hard for the reader to track K96, T12, and N09 when their definitions are embedded pages before they are referenced.  I suggest including a small table with these labels, what they are, and their full citation so readers can quickly understand what is meant.
    - Unfortunately, we could not find any suitable place to put the suggested table. However, we made sure that all of the K96, T12 and N09 abbreviations are hyperlinked. In the PDF version, the reader can click on them to see the actual citation.

When re-using figures from a previous publication, you may need copyright permission from the journal source (e.g., Figure 1 was published by the Astronomical Journal).
    - Thank you for reminding us about the copyright permission. We checked the policy of the Astronomical Journal and for copyright purposes the permission of the author is enough. The sole author of the T12 paper is a co-author to the current paper and we have his permission to use the figure.
